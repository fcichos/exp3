%% Generated by Sphinx.
\def\sphinxdocclass{report}
\documentclass[letterpaper,10pt,english]{sphinxmanual}
\ifdefined\pdfpxdimen
   \let\sphinxpxdimen\pdfpxdimen\else\newdimen\sphinxpxdimen
\fi \sphinxpxdimen=.75bp\relax
\ifdefined\pdfimageresolution
    \pdfimageresolution= \numexpr \dimexpr1in\relax/\sphinxpxdimen\relax
\fi
%% let collapsible pdf bookmarks panel have high depth per default
\PassOptionsToPackage{bookmarksdepth=5}{hyperref}

\PassOptionsToPackage{warn}{textcomp}
\usepackage[utf8]{inputenc}
\ifdefined\DeclareUnicodeCharacter
% support both utf8 and utf8x syntaxes
  \ifdefined\DeclareUnicodeCharacterAsOptional
    \def\sphinxDUC#1{\DeclareUnicodeCharacter{"#1}}
  \else
    \let\sphinxDUC\DeclareUnicodeCharacter
  \fi
  \sphinxDUC{00A0}{\nobreakspace}
  \sphinxDUC{2500}{\sphinxunichar{2500}}
  \sphinxDUC{2502}{\sphinxunichar{2502}}
  \sphinxDUC{2514}{\sphinxunichar{2514}}
  \sphinxDUC{251C}{\sphinxunichar{251C}}
  \sphinxDUC{2572}{\textbackslash}
\fi
\usepackage{cmap}
\usepackage[T1]{fontenc}
\usepackage{amsmath,amssymb,amstext}
\usepackage{babel}



\usepackage{tgtermes}
\usepackage{tgheros}
\renewcommand{\ttdefault}{txtt}



\usepackage[Bjarne]{fncychap}
\usepackage{sphinx}

\fvset{fontsize=auto}
\usepackage{geometry}


% Include hyperref last.
\usepackage{hyperref}
% Fix anchor placement for figures with captions.
\usepackage{hypcap}% it must be loaded after hyperref.
% Set up styles of URL: it should be placed after hyperref.
\urlstyle{same}

\addto\captionsenglish{\renewcommand{\contentsname}{Course Information:}}

\usepackage{sphinxmessages}
\setcounter{tocdepth}{1}


% Jupyter Notebook code cell colors
\definecolor{nbsphinxin}{HTML}{307FC1}
\definecolor{nbsphinxout}{HTML}{BF5B3D}
\definecolor{nbsphinx-code-bg}{HTML}{F5F5F5}
\definecolor{nbsphinx-code-border}{HTML}{E0E0E0}
\definecolor{nbsphinx-stderr}{HTML}{FFDDDD}
% ANSI colors for output streams and traceback highlighting
\definecolor{ansi-black}{HTML}{3E424D}
\definecolor{ansi-black-intense}{HTML}{282C36}
\definecolor{ansi-red}{HTML}{E75C58}
\definecolor{ansi-red-intense}{HTML}{B22B31}
\definecolor{ansi-green}{HTML}{00A250}
\definecolor{ansi-green-intense}{HTML}{007427}
\definecolor{ansi-yellow}{HTML}{DDB62B}
\definecolor{ansi-yellow-intense}{HTML}{B27D12}
\definecolor{ansi-blue}{HTML}{208FFB}
\definecolor{ansi-blue-intense}{HTML}{0065CA}
\definecolor{ansi-magenta}{HTML}{D160C4}
\definecolor{ansi-magenta-intense}{HTML}{A03196}
\definecolor{ansi-cyan}{HTML}{60C6C8}
\definecolor{ansi-cyan-intense}{HTML}{258F8F}
\definecolor{ansi-white}{HTML}{C5C1B4}
\definecolor{ansi-white-intense}{HTML}{A1A6B2}
\definecolor{ansi-default-inverse-fg}{HTML}{FFFFFF}
\definecolor{ansi-default-inverse-bg}{HTML}{000000}

% Define an environment for non-plain-text code cell outputs (e.g. images)
\makeatletter
\newenvironment{nbsphinxfancyoutput}{%
    % Avoid fatal error with framed.sty if graphics too long to fit on one page
    \let\sphinxincludegraphics\nbsphinxincludegraphics
    \nbsphinx@image@maxheight\textheight
    \advance\nbsphinx@image@maxheight -2\fboxsep   % default \fboxsep 3pt
    \advance\nbsphinx@image@maxheight -2\fboxrule  % default \fboxrule 0.4pt
    \advance\nbsphinx@image@maxheight -\baselineskip
\def\nbsphinxfcolorbox{\spx@fcolorbox{nbsphinx-code-border}{white}}%
\def\FrameCommand{\nbsphinxfcolorbox\nbsphinxfancyaddprompt\@empty}%
\def\FirstFrameCommand{\nbsphinxfcolorbox\nbsphinxfancyaddprompt\sphinxVerbatim@Continues}%
\def\MidFrameCommand{\nbsphinxfcolorbox\sphinxVerbatim@Continued\sphinxVerbatim@Continues}%
\def\LastFrameCommand{\nbsphinxfcolorbox\sphinxVerbatim@Continued\@empty}%
\MakeFramed{\advance\hsize-\width\@totalleftmargin\z@\linewidth\hsize\@setminipage}%
\lineskip=1ex\lineskiplimit=1ex\raggedright%
}{\par\unskip\@minipagefalse\endMakeFramed}
\makeatother
\newbox\nbsphinxpromptbox
\def\nbsphinxfancyaddprompt{\ifvoid\nbsphinxpromptbox\else
    \kern\fboxrule\kern\fboxsep
    \copy\nbsphinxpromptbox
    \kern-\ht\nbsphinxpromptbox\kern-\dp\nbsphinxpromptbox
    \kern-\fboxsep\kern-\fboxrule\nointerlineskip
    \fi}
\newlength\nbsphinxcodecellspacing
\setlength{\nbsphinxcodecellspacing}{0pt}

% Define support macros for attaching opening and closing lines to notebooks
\newsavebox\nbsphinxbox
\makeatletter
\newcommand{\nbsphinxstartnotebook}[1]{%
    \par
    % measure needed space
    \setbox\nbsphinxbox\vtop{{#1\par}}
    % reserve some space at bottom of page, else start new page
    \needspace{\dimexpr2.5\baselineskip+\ht\nbsphinxbox+\dp\nbsphinxbox}
    % mimick vertical spacing from \section command
      \addpenalty\@secpenalty
      \@tempskipa 3.5ex \@plus 1ex \@minus .2ex\relax
      \addvspace\@tempskipa
      {\Large\@tempskipa\baselineskip
             \advance\@tempskipa-\prevdepth
             \advance\@tempskipa-\ht\nbsphinxbox
             \ifdim\@tempskipa>\z@
               \vskip \@tempskipa
             \fi}
    \unvbox\nbsphinxbox
    % if notebook starts with a \section, prevent it from adding extra space
    \@nobreaktrue\everypar{\@nobreakfalse\everypar{}}%
    % compensate the parskip which will get inserted by next paragraph
    \nobreak\vskip-\parskip
    % do not break here
    \nobreak
}% end of \nbsphinxstartnotebook

\newcommand{\nbsphinxstopnotebook}[1]{%
    \par
    % measure needed space
    \setbox\nbsphinxbox\vbox{{#1\par}}
    \nobreak % it updates page totals
    \dimen@\pagegoal
    \advance\dimen@-\pagetotal \advance\dimen@-\pagedepth
    \advance\dimen@-\ht\nbsphinxbox \advance\dimen@-\dp\nbsphinxbox
    \ifdim\dimen@<\z@
      % little space left
      \unvbox\nbsphinxbox
      \kern-.8\baselineskip
      \nobreak\vskip\z@\@plus1fil
      \penalty100
      \vskip\z@\@plus-1fil
      \kern.8\baselineskip
    \else
      \unvbox\nbsphinxbox
    \fi
}% end of \nbsphinxstopnotebook

% Ensure height of an included graphics fits in nbsphinxfancyoutput frame
\newdimen\nbsphinx@image@maxheight % set in nbsphinxfancyoutput environment
\newcommand*{\nbsphinxincludegraphics}[2][]{%
    \gdef\spx@includegraphics@options{#1}%
    \setbox\spx@image@box\hbox{\includegraphics[#1,draft]{#2}}%
    \in@false
    \ifdim \wd\spx@image@box>\linewidth
      \g@addto@macro\spx@includegraphics@options{,width=\linewidth}%
      \in@true
    \fi
    % no rotation, no need to worry about depth
    \ifdim \ht\spx@image@box>\nbsphinx@image@maxheight
      \g@addto@macro\spx@includegraphics@options{,height=\nbsphinx@image@maxheight}%
      \in@true
    \fi
    \ifin@
      \g@addto@macro\spx@includegraphics@options{,keepaspectratio}%
    \fi
    \setbox\spx@image@box\box\voidb@x % clear memory
    \expandafter\includegraphics\expandafter[\spx@includegraphics@options]{#2}%
}% end of "\MakeFrame"-safe variant of \sphinxincludegraphics
\makeatother

\makeatletter
\renewcommand*\sphinx@verbatim@nolig@list{\do\'\do\`}
\begingroup
\catcode`'=\active
\let\nbsphinx@noligs\@noligs
\g@addto@macro\nbsphinx@noligs{\let'\PYGZsq}
\endgroup
\makeatother
\renewcommand*\sphinxbreaksbeforeactivelist{\do\<\do\"\do\'}
\renewcommand*\sphinxbreaksafteractivelist{\do\.\do\,\do\:\do\;\do\?\do\!\do\/\do\>\do\-}
\makeatletter
\fvset{codes*=\sphinxbreaksattexescapedchars\do\^\^\let\@noligs\nbsphinx@noligs}
\makeatother



\title{Experimental Physics 3}
\date{Oct 10, 2022}
\release{1}
\author{Frank Cichos}
\newcommand{\sphinxlogo}{\vbox{}}
\renewcommand{\releasename}{Release}
\makeindex
\begin{document}

\pagestyle{empty}
\sphinxmaketitle
\pagestyle{plain}
\sphinxtableofcontents
\pagestyle{normal}
\phantomsection\label{\detokenize{index::doc}}
\begin{figure}[htbp]
\centering

\noindent\sphinxincludegraphics[width=8711\sphinxpxdimen,height=1893\sphinxpxdimen]{{CompSoft_banner}.png}
\end{figure}



\sphinxAtStartPar
In this Experimental Physics 3 course, we will dive into to basic experiments and mathematical descriptions related to light propagation, electromagnetic waves and its material counter part of matter waves. In particular we will have a look at
\begin{itemize}
\item {} 
\sphinxAtStartPar
Geometrical Optics

\item {} 
\sphinxAtStartPar
Wave Optics1

\item {} 
\sphinxAtStartPar
Electromagnetic Waves

\item {} 
\sphinxAtStartPar
Matter Waves and Quantum Mechanics

\end{itemize}

\sphinxAtStartPar
The fields of optics and quantum mechanics are nowadays very active research areas with a dynamically devloping field of optical technologies, high resolution microscopy and quantum information. All this builds on the fundations that are tackled in this course. To back up the lectures I further recommend \sphinxstylestrong{books} that are listed in the \sphinxstylestrong{resources section} of the website.


\chapter{This Website}
\label{\detokenize{course-info/website:this-website}}\label{\detokenize{course-info/website::doc}}
\sphinxAtStartPar
This website will contain all the information that are required for our \sphinxstylestrong{Experimental Physics 3} course. It is not yet complete, but it will be updated each week and you will find new lectures every week.
All of the lectures will be made available as Jupyter notebooks and videos.
You will be guided from here to several resources that you can use to learn programming in Python.


\chapter{Course Schedule}
\label{\detokenize{course-info/schedule:course-schedule}}\label{\detokenize{course-info/schedule::doc}}
\sphinxAtStartPar
The course will be held in two weekly lectures, starting \sphinxstylestrong{10.10.2020} in person.

\begin{DUlineblock}{0em}
\item[] \sphinxstylestrong{Monday 11:15 am \textendash{} 12:45 pm}
\item[] \sphinxstylestrong{Thursday 11:15 am \textendash{} 12:45 pm}
\end{DUlineblock}

\sphinxAtStartPar
The course and the material is available online on this website. You may come back to study whenever it is suitable for you.

\sphinxAtStartPar
The seminar will be held online by Tobias and Markus. Please also use the discussion forum to ask questions and leave comments.


\chapter{Assignments}
\label{\detokenize{course-info/assignments:assignments}}\label{\detokenize{course-info/assignments::doc}}
\sphinxAtStartPar
All of the assignments of the lecture will be handled in moodle. An assigment will be handed out every week starting the 17.October 2022 in the afternoon. The assignment will be due the week after before the lecture until 11:00 o’clock. All assignments will be corrected and the solution to the individual problems will be discussed in the seminar one week later.

\sphinxAtStartPar
\sphinxstylestrong{To take part in the final exam you need 50\% (sharp!) of the total point possible.}

\sphinxAtStartPar
The video below tells you how to access the assignments and how to submit the solutions.




\chapter{Exams}
\label{\detokenize{course-info/exam:exams}}\label{\detokenize{course-info/exam::doc}}
\sphinxAtStartPar
This course will end with a \sphinxstylestrong{written exam} of \sphinxstylestrong{180 min} duration. To take part in the exam you need to qualify with 50 \% of the possible points of the exercises handed out weekly.


\chapter{Resources}
\label{\detokenize{course-info/resources:resources}}\label{\detokenize{course-info/resources::doc}}

\section{Books}
\label{\detokenize{course-info/resources:books}}
\sphinxAtStartPar
The course will be mainly build on a number of excellent books on electrodynamics and optics as well as on the basics of quantum mechanics.
\begin{enumerate}
\sphinxsetlistlabels{\arabic}{enumi}{enumii}{}{.}%
\item {} 
\sphinxAtStartPar
\sphinxstylestrong{Optics and Electrodynamics}
\begin{itemize}
\item {} 
\sphinxAtStartPar
Demtröder: Electrodynamics and Optics, Springer, 2019

\item {} 
\sphinxAtStartPar
Saleh/Teich: Fundamentals of Photonics, Wiley, 2007

\item {} 
\sphinxAtStartPar
Jackson: Classical Electrodynamics, Wiley, 1998

\item {} 
\sphinxAtStartPar
Hecht: Optics, Pearson, 2016

\item {} 
\sphinxAtStartPar
My handwritten lecture notes for download \sphinxcode{\sphinxupquote{pdf}}

\end{itemize}

\item {} 
\sphinxAtStartPar
\sphinxstylestrong{Quantum Mechanics}
\begin{itemize}
\item {} 
\sphinxAtStartPar
Demtröder: Atoms, Molecules and Photons, Springer, 2010

\item {} 
\sphinxAtStartPar
Haken, Wolf: The Physics of Atoms and Quanta: Introduction to Experiments and Theory, Springer, 2005

\item {} 
\sphinxAtStartPar
Harnwel, Livingood: Experimental Atomic Physics, McGraw\sphinxhyphen{}Hill Book Company, Inc, 1933

\end{itemize}

\end{enumerate}


\section{Molecular Nanophotonics Group}
\label{\detokenize{course-info/resources:molecular-nanophotonics-group}}
\sphinxAtStartPar
Besides the books, you may also want to have a look at the following websites maintained by the group
\begin{itemize}
\item {} 
\sphinxAtStartPar
\sphinxhref{https://home.uni-leipzig.de/~physik/sites/mona/}{Molecular Nanophotonics Group Website}

\item {} 
\sphinxAtStartPar
\sphinxhref{https://fcichos.github.com/website/}{Computer\sphinxhyphen{}based Physical Modeling Website @ MONA}

\end{itemize}


\chapter{Instructors}
\label{\detokenize{course-info/instructor:instructors}}\label{\detokenize{course-info/instructor::doc}}\begin{itemize}
\item {} 
\sphinxAtStartPar
Prof. Dr. Frank Cichos
\begin{itemize}
\item {} 
\sphinxAtStartPar
Linnéstr. 5, 04103 Leipzig

\item {} 
\sphinxAtStartPar
Office: 322

\item {} 
\sphinxAtStartPar
Phone: +0341 97 32571

\item {} 
\sphinxAtStartPar
Email: \sphinxstyleemphasis{lastname@physik.uni\sphinxhyphen{}leipzig.de}

\end{itemize}

\end{itemize}

\begin{DUlineblock}{0em}
\item[] 
\end{DUlineblock}
\begin{itemize}
\item {} 
\sphinxAtStartPar
Dr. Tobias Thalheim
\begin{itemize}
\item {} 
\sphinxAtStartPar
Linnéstr. 5, 04103 Leipzig

\item {} 
\sphinxAtStartPar
Office: 318

\item {} 
\sphinxAtStartPar
Phone: +0341 97 32570

\item {} 
\sphinxAtStartPar
Email: \sphinxstyleemphasis{firstname.lastname@physik.uni\sphinxhyphen{}leipzig.de}

\end{itemize}

\end{itemize}

\begin{DUlineblock}{0em}
\item[] 
\end{DUlineblock}
\begin{itemize}
\item {} 
\sphinxAtStartPar
Dr. Markus Anton
\begin{itemize}
\item {} 
\sphinxAtStartPar
Linnéstr. 5, 04103 Leipzig

\item {} 
\sphinxAtStartPar
Office: 102a

\item {} 
\sphinxAtStartPar
Phone: +0341 97 32575

\item {} 
\sphinxAtStartPar
Email: \sphinxstyleemphasis{firstname.lastname@studserv.uni\sphinxhyphen{}leipzig.de}

\end{itemize}

\end{itemize}


\chapter{Overview}
\label{\detokenize{lectures/Intro/overview:overview}}\label{\detokenize{lectures/Intro/overview::doc}}
\sphinxAtStartPar
The Experimental Physics 3 course is introducing you to topics related to electromagnetic waves, optics and matter waves.
We will cover experiments, some basic mathematical description and also interactive visualizations in our lecture. Below
you find the planned contents of our course. Besides the physical contents, this website also contains a number of additional
interactive features. These features are provided by so\sphinxhyphen{}called Jupyter notebooks and the contained Python code. If you want to know more about those possibilities, have a look at the sections on Jupyter notebooks in the Course Introduction.



\begin{DUlineblock}{0em}
\item[] 
\end{DUlineblock}


\section{Lecture Contents}
\label{\detokenize{lectures/Intro/overview:lecture-contents}}\begin{enumerate}
\sphinxsetlistlabels{\arabic}{enumi}{enumii}{}{.}%
\item {} \begin{description}
\item[{Ray Optics}] \leavevmode
\begin{DUlineblock}{0em}
\item[] 1.1.Reflection
\item[] 1.2.Refraction, Total internal reflection, Rainbow challenge
\item[] 1.3.Mirrors, Lenses, Prisms
\item[] 1.4.Optical instruments
\item[]
\begin{DUlineblock}{\DUlineblockindent}
\item[] 1.4.1.Telescope
\item[] 1.4.2.Microscope
\end{DUlineblock}
\item[] 1.5.Dispersion
\item[] 1.6.Imaging errors
\item[]
\begin{DUlineblock}{\DUlineblockindent}
\item[] 1.6.1.Spherical aberration
\item[] 1.6.2.Coma
\item[] 1.6.3.Astigmatism
\item[] 1.6.4.Chromatic aberration
\end{DUlineblock}
\end{DUlineblock}

\end{description}

\item {} \begin{description}
\item[{Wave Optics}] \leavevmode
\begin{DUlineblock}{0em}
\item[] 2.1.Wave equation
\item[]
\begin{DUlineblock}{\DUlineblockindent}
\item[] 2.1.1.Plane waves
\item[] 2.1.2.Spherical waves
\end{DUlineblock}
\item[] 2.2.Interference
\item[]
\begin{DUlineblock}{\DUlineblockindent}
\item[] 2.2.1.Coherence
\item[] 2.2.2.Interferometers
\end{DUlineblock}
\item[] 2.3.Huygens principle
\item[]
\begin{DUlineblock}{\DUlineblockindent}
\item[] 2.3.1.Diffraction
\item[] 2.3.2.Single and double slit
\item[] 2.3.3.Diffraction grating
\item[] 2.3.4.Optical resolution
\end{DUlineblock}
\end{DUlineblock}

\end{description}

\item {} \begin{description}
\item[{Electromagnetic Waves}] \leavevmode
\begin{DUlineblock}{0em}
\item[] 3.1.Electromagnetic spectrum
\item[] 3.2.Plane and spherical electromagnetic waves
\item[] 3.3.Energy transport and Poynting vector
\item[] 3.4.Polarization
\item[] 3.5.Reflection and transmission
\item[] 3.6.Total internal reflection
\item[] 3.7.Fresnel formulas
\item[] 3.8.Hertz dipole
\end{DUlineblock}

\end{description}

\item {} \begin{description}
\item[{Foundations of Quantum Physics}] \leavevmode
\begin{DUlineblock}{0em}
\item[] 4.1.Particle properties of light
\item[]
\begin{DUlineblock}{\DUlineblockindent}
\item[] 4.1.1.Photo effect
\item[] 4.1.2.Black body radiation
\item[] 4.1.3.Photon gas
\item[] 4.1.4.Planck’s radiation law
\end{DUlineblock}
\item[] 4.2.Structure of matter
\item[]
\begin{DUlineblock}{\DUlineblockindent}
\item[] 4.2.1.Thomson model of the atom
\item[] 4.2.2.Rutherford scattering
\item[] 4.2.3.Rutherford and Bohr atom model
\end{DUlineblock}
\item[] 4.3.Matter waves
\item[]
\begin{DUlineblock}{\DUlineblockindent}
\item[] 4.3.1.Heisenberg uncertainty relation
\item[] 4.3.2.Wave function
\item[] 4.3.3.Probability interpretation of the wave function
\item[] 4.3.4.Schrödinger equation
\item[] 4.3.5.Quantum states
\item[] 4.3.6.Potential box
\item[] 4.3.7.Harmonic oscillator
\item[] 4.3.8.Tunneling
\item[] 4.3.9.Correspondence principle
\end{DUlineblock}
\end{DUlineblock}

\end{description}

\end{enumerate}

\sphinxAtStartPar
The following section was created from \sphinxcode{\sphinxupquote{/home/lectures/exp3/source/notebooks/Intro/Introduction2Jupyter.ipynb}}.


\chapter{Introduction to Jupyter}
\label{\detokenize{notebooks/Intro/Introduction2Jupyter:Introduction-to-Jupyter}}\label{\detokenize{notebooks/Intro/Introduction2Jupyter::doc}}

\section{What is Jupyter Notebook?}
\label{\detokenize{notebooks/Intro/Introduction2Jupyter:What-is-Jupyter-Notebook?}}
\sphinxAtStartPar
A Jupyter Notebook is a web browser based \sphinxstylestrong{interactive computing environment} that enables users to create documents that include code to be executed, results from the executed code such as plots and images, and finally also an additional documentation in form of markdown text including equations in LaTeX.

\sphinxAtStartPar
These documents provide a \sphinxstylestrong{complete and self\sphinxhyphen{}contained record of a computation} that can be converted to various formats and shared with others using email, version control systems (like git/\sphinxhref{https://github.com}{GitHub}) or \sphinxhref{http://nbviewer.jupyter.org}{nbviewer.jupyter.org}.

\sphinxAtStartPar
The Jupyter Notebook combines three components:
\begin{itemize}
\item {} 
\sphinxAtStartPar
\sphinxstylestrong{Notebook editor}: An interactive application for writing and running code interactively and editing notebook documents. If you run Jupyter on desktop, you will be using Jupyter’s web application.

\item {} 
\sphinxAtStartPar
\sphinxstylestrong{Kernels}: Separate processes started by Jupyter on your server, that runs users’ code in a given language and returns output back to the notebook web application. The kernel also handles things like computations for interactive widgets, tab completion and introspection.

\item {} 
\sphinxAtStartPar
\sphinxstylestrong{Notebook documents}: Self\sphinxhyphen{}contained documents that contain a representation of all content visible in the notebook editor, including inputs and outputs of the computations, markdown text, equations, images, and rich media representations of objects. Each notebook document has its own kernel.

\end{itemize}


\section{Notebook editor}
\label{\detokenize{notebooks/Intro/Introduction2Jupyter:Notebook-editor}}
\sphinxAtStartPar
The Notebook editor is a web application running in your browser. It enables you to
\begin{itemize}
\item {} 
\sphinxAtStartPar
\sphinxstylestrong{Edit code} in individual cells

\item {} 
\sphinxAtStartPar
\sphinxstylestrong{Run code} in individuall cells in arbitrary order and display results of the computation in various formats (HTML, LaTeX, PNG, SVG, PDF)

\item {} 
\sphinxAtStartPar
Create and use \sphinxstylestrong{interactive JavaScript widgets}, which bind interactive user interface controls and visualizations to reactive kernel side computations.

\item {} 
\sphinxAtStartPar
Add \sphinxstylestrong{documentation text} using \sphinxhref{https://daringfireball.net/projects/markdown/}{Markdown} markup language, including LaTeX equations

\end{itemize}


\section{Kernels}
\label{\detokenize{notebooks/Intro/Introduction2Jupyter:Kernels}}
\sphinxAtStartPar
The Jupyter notebook is not bound to any specific programming language, but can be used for almost any type of language. Each Jupyter notebook starts a server application that is connected to a kernel that runs the code in the notebook. This kernel is dedicated to a specific programming language. Thus, installing different kernels \sphinxhref{https://github.com/jupyter/jupyter/wiki/Jupyter-kernels}{100+ languages} allows you to execute code in \sphinxstylestrong{Python}, \sphinxstylestrong{Julia}, \sphinxstylestrong{R}, \sphinxstylestrong{Ruby}, \sphinxstylestrong{Haskell},
\sphinxstylestrong{Scala}, and many others.

\sphinxAtStartPar
Yet, the default kernel runs Python code. The notebook provides a simple way for users to pick which of these kernels is used for a given notebook. Each of these kernels communicate with the notebook editor using JSON over the ZeroMQ/WebSockets message protocol that is described \sphinxhref{https://jupyter-client.readthedocs.io/en/latest/messaging.html\#messaging}{here}. Most users do not need to know about these details, but it helps to understand that “kernels run code”.


\section{Notebook documents}
\label{\detokenize{notebooks/Intro/Introduction2Jupyter:Notebook-documents}}
\sphinxAtStartPar
Notebook documents, or notebooks, contain the \sphinxstylestrong{inputs and outputs} of an interactive session as well as \sphinxstylestrong{documentation text} that accompanies the code but is not meant for execution.

\sphinxAtStartPar
A notebook is just a \sphinxstylestrong{file on your server’s filesystem with a \textasciigrave{}\textasciigrave{}.ipynb\textasciigrave{}\textasciigrave{} extension}. This allows you to share your notebook easily.

\sphinxAtStartPar
Notebooks consist of a \sphinxstylestrong{linear sequence of cells}. There are three basic cell types:
\begin{itemize}
\item {} 
\sphinxAtStartPar
\sphinxstylestrong{Code cells:} Input and output of live code that is run in the kernel.

\item {} 
\sphinxAtStartPar
\sphinxstylestrong{Markdown cells:} Narrative text with embedded LaTeX equations.

\item {} 
\sphinxAtStartPar
\sphinxstylestrong{Raw cells:} Unformatted text that is included, without modification, when notebooks are converted to different formats using \sphinxcode{\sphinxupquote{nbconvert}}.

\end{itemize}

\sphinxAtStartPar
Internally, notebook documents are \sphinxhref{https://en.wikipedia.org/wiki/JSON}{JSON}\sphinxstylestrong{text files} with \sphinxstylestrong{binary data encoded in}\sphinxhref{http://en.wikipedia.org/wiki/Base64}{base64}. This allows them to be \sphinxstylestrong{read and manipulated programmatically} by any programming language.

\sphinxAtStartPar
\sphinxstylestrong{Notebooks can be exported} to different static formats including HTML, reStructeredText, LaTeX, PDF, and slide shows (\sphinxhref{http://lab.hakim.se/reveal-js/}{reveal.js}) using Jupyter’s \sphinxcode{\sphinxupquote{nbconvert}} utility.

\sphinxAtStartPar
Furthermore, any notebook document available from a \sphinxstylestrong{public URL on or GitHub can be shared} via \sphinxhref{http://nbviewer.jupyter.org}{nbviewer}. This service loads the notebook document from the URL and renders it as a static web page. The resulting web page may thus be shared with others \sphinxstylestrong{without their needing to install the Jupyter Notebook}.

\sphinxAtStartPar
The following section was created from \sphinxcode{\sphinxupquote{/home/lectures/exp3/source/notebooks/Intro/NotebookEditor.ipynb}}.


\chapter{Notebook editor}
\label{\detokenize{notebooks/Intro/NotebookEditor:Notebook-editor}}\label{\detokenize{notebooks/Intro/NotebookEditor::doc}}
\noindent\sphinxincludegraphics[width=778\sphinxpxdimen,height=478\sphinxpxdimen]{{notebook}.png}

\sphinxAtStartPar
A Jupyter Notebook provides an interface with essentially two modes
\begin{itemize}
\item {} 
\sphinxAtStartPar
\sphinxstylestrong{edit mode} the mode where you edit a cell’s content.

\item {} 
\sphinxAtStartPar
\sphinxstylestrong{command mode} the mode where you execute the cells content.

\end{itemize}

\sphinxAtStartPar
In the more advanced version of JupyterLab, which we are using on myBinder, this will look like that


\section{Edit mode}
\label{\detokenize{notebooks/Intro/NotebookEditor:Edit-mode}}
\sphinxAtStartPar
Edit mode is indicated by a blue cell border and a prompt showing in the editor area:

\noindent\sphinxincludegraphics[width=526\sphinxpxdimen,height=45\sphinxpxdimen]{{edit_mode}.png}

\sphinxAtStartPar
When a cell is in edit mode, you can type into the cell, like a normal text editor.


\section{Command mode}
\label{\detokenize{notebooks/Intro/NotebookEditor:Command-mode}}
\sphinxAtStartPar
Command mode is indicated by a grey cell border:

\noindent\sphinxincludegraphics[width=521\sphinxpxdimen,height=46\sphinxpxdimen]{{command_mode}.png}


\section{Keyboard navigation}
\label{\detokenize{notebooks/Intro/NotebookEditor:Keyboard-navigation}}
\sphinxAtStartPar
If you have a hardware keyboard connected to your iOS device, you can use Jupyter keyboard shortcuts. The modal user interface of the Jupyter Notebook has been optimized for efficient keyboard usage. This is made possible by having two different sets of keyboard shortcuts: one set that is active in edit mode and another in command mode.

\sphinxAtStartPar
In edit mode, most of the keyboard is dedicated to typing into the cell’s editor. Thus, in edit mode there are relatively few shortcuts. In command mode, the entire keyboard is available for shortcuts, so there are many more. Most important ones are:
\begin{enumerate}
\sphinxsetlistlabels{\arabic}{enumi}{enumii}{}{.}%
\item {} 
\sphinxAtStartPar
Switch command and edit mods: \sphinxcode{\sphinxupquote{Enter}} for edit mode, and \sphinxcode{\sphinxupquote{Esc}} or \sphinxcode{\sphinxupquote{Control}} for command mode.

\item {} 
\sphinxAtStartPar
Basic navigation: \sphinxcode{\sphinxupquote{↑}}/\sphinxcode{\sphinxupquote{k}}, \sphinxcode{\sphinxupquote{↓}}/\sphinxcode{\sphinxupquote{j}}

\item {} 
\sphinxAtStartPar
Run or render currently selected cell: \sphinxcode{\sphinxupquote{Shift}}+\sphinxcode{\sphinxupquote{Enter}} or \sphinxcode{\sphinxupquote{Control}}+\sphinxcode{\sphinxupquote{Enter}}

\item {} 
\sphinxAtStartPar
Saving the notebook: \sphinxcode{\sphinxupquote{s}}

\item {} 
\sphinxAtStartPar
Change cell types: \sphinxcode{\sphinxupquote{y}} to make it a \sphinxstylestrong{code} cell, \sphinxcode{\sphinxupquote{m}} for \sphinxstylestrong{markdown} and \sphinxcode{\sphinxupquote{r}} for \sphinxstylestrong{raw}

\item {} 
\sphinxAtStartPar
Inserting new cells: \sphinxcode{\sphinxupquote{a}} to \sphinxstylestrong{insert above}, \sphinxcode{\sphinxupquote{b}} to \sphinxstylestrong{insert below}

\item {} 
\sphinxAtStartPar
Manipulating cells using pasteboard: \sphinxcode{\sphinxupquote{x}} for \sphinxstylestrong{cut}, \sphinxcode{\sphinxupquote{c}} for \sphinxstylestrong{copy}, \sphinxcode{\sphinxupquote{v}} for \sphinxstylestrong{paste}, \sphinxcode{\sphinxupquote{d}} for \sphinxstylestrong{delete} and \sphinxcode{\sphinxupquote{z}} for \sphinxstylestrong{undo delete}

\item {} 
\sphinxAtStartPar
Kernel operations: \sphinxcode{\sphinxupquote{i}} to \sphinxstylestrong{interrupt} and \sphinxcode{\sphinxupquote{0}} to \sphinxstylestrong{restart}

\end{enumerate}


\section{Running code}
\label{\detokenize{notebooks/Intro/NotebookEditor:Running-code}}
\sphinxAtStartPar
Code cells allow you to enter and run code. Run a code cell by pressing the \sphinxcode{\sphinxupquote{▶︎}} button in the bottom\sphinxhyphen{}right panel, or \sphinxcode{\sphinxupquote{Control}}+\sphinxcode{\sphinxupquote{Enter}} on your hardware keyboard.

\begin{sphinxuseclass}{nbinput}
\begin{sphinxuseclass}{nblast}
{
\sphinxsetup{VerbatimColor={named}{nbsphinx-code-bg}}
\sphinxsetup{VerbatimBorderColor={named}{nbsphinx-code-border}}
\begin{sphinxVerbatim}[commandchars=\\\{\}]
\llap{\color{nbsphinxin}[1]:\,\hspace{\fboxrule}\hspace{\fboxsep}}\PYG{n}{v} \PYG{o}{=} \PYG{l+m+mi}{10}
\end{sphinxVerbatim}
}

\end{sphinxuseclass}
\end{sphinxuseclass}
\begin{sphinxuseclass}{nbinput}
{
\sphinxsetup{VerbatimColor={named}{nbsphinx-code-bg}}
\sphinxsetup{VerbatimBorderColor={named}{nbsphinx-code-border}}
\begin{sphinxVerbatim}[commandchars=\\\{\}]
\llap{\color{nbsphinxin}[5]:\,\hspace{\fboxrule}\hspace{\fboxsep}}\PYG{n+nb}{print}\PYG{p}{(}\PYG{n}{v}\PYG{p}{)}
\end{sphinxVerbatim}
}

\end{sphinxuseclass}
\begin{sphinxuseclass}{nboutput}
\begin{sphinxuseclass}{nblast}
{

\kern-\sphinxverbatimsmallskipamount\kern-\baselineskip
\kern+\FrameHeightAdjust\kern-\fboxrule
\vspace{\nbsphinxcodecellspacing}

\sphinxsetup{VerbatimColor={named}{white}}
\sphinxsetup{VerbatimBorderColor={named}{nbsphinx-code-border}}
\begin{sphinxuseclass}{output_area}
\begin{sphinxuseclass}{}


\begin{sphinxVerbatim}[commandchars=\\\{\}]
100
\end{sphinxVerbatim}



\end{sphinxuseclass}
\end{sphinxuseclass}
}

\end{sphinxuseclass}
\end{sphinxuseclass}
\sphinxAtStartPar
There are a couple of keyboard shortcuts for running code:
\begin{itemize}
\item {} 
\sphinxAtStartPar
\sphinxcode{\sphinxupquote{Control}}+\sphinxcode{\sphinxupquote{Enter}} runs the current cell and enters command mode.

\item {} 
\sphinxAtStartPar
\sphinxcode{\sphinxupquote{Shift}}+\sphinxcode{\sphinxupquote{Enter}} runs the current cell and moves selection to the one below.

\item {} 
\sphinxAtStartPar
\sphinxcode{\sphinxupquote{Option}}+\sphinxcode{\sphinxupquote{Enter}} runs the current cell and inserts a new one below.

\end{itemize}


\section{Managing the kernel}
\label{\detokenize{notebooks/Intro/NotebookEditor:Managing-the-kernel}}
\sphinxAtStartPar
Code is run in a separate process called the \sphinxstylestrong{kernel}, which can be interrupted or restarted. You can see the kernel indicator in the top\sphinxhyphen{}right corner reporting the current kernel state: \sphinxcode{\sphinxupquote{⚪︎}} means the kernel is \sphinxstylestrong{ready} to execute code, and \sphinxcode{\sphinxupquote{⚫︎}} means the kernel is currently \sphinxstylestrong{busy}. Tapping kernel indicator will open the \sphinxstylestrong{kernel menu}, where you can reconnect, interrupt or restart the kernel.

\sphinxAtStartPar
Try running the following cell — kernel indicator will switch from \sphinxcode{\sphinxupquote{⚪︎}} to \sphinxcode{\sphinxupquote{⚫︎}}, i.e., reporting the kernel as “busy”. This means that you will not be able to run any new cells until the current execution finishes, or until the kernel is interrupted. You can then go to the kernel menu by tapping the kernel indicator and select “Interrupt”.

\sphinxAtStartPar
The following section was created from \sphinxcode{\sphinxupquote{/home/lectures/exp3/source/notebooks/Intro/EditCells.ipynb}}.


\chapter{Entering code}
\label{\detokenize{notebooks/Intro/EditCells:Entering-code}}\label{\detokenize{notebooks/Intro/EditCells::doc}}
\sphinxAtStartPar
Entering code is pretty easy. You just have to click into a cell and type the commands you want to type. If you have multiple lines of code, just press \sphinxstylestrong{enter} at the end of the line and start a new one.
\begin{itemize}
\item {} 
\sphinxAtStartPar
\sphinxstylestrong{code blocks} Python identifies blocks of codes belonging together by its identation. This will become important if you write longer code in a cell later. To indent the block, you may use either \sphinxstyleemphasis{whitespaces} or \sphinxstyleemphasis{tabs}.

\item {} 
\sphinxAtStartPar
\sphinxstylestrong{comments} Comments can be added to annotate the code, such that you or someone else can understand the code.
\begin{itemize}
\item {} 
\sphinxAtStartPar
Comments in a single line are started with the \sphinxcode{\sphinxupquote{\#}} character in front of the comment.

\item {} 
\sphinxAtStartPar
Comments over multiple lines can be started with \sphinxcode{\sphinxupquote{\textquotesingle{}\textquotesingle{}\textquotesingle{}}}and end with the same \sphinxcode{\sphinxupquote{\textquotesingle{}\textquotesingle{}\textquotesingle{}}}. They are used as \sphinxcode{\sphinxupquote{docstrings}} to provide a help text.

\end{itemize}

\end{itemize}

\begin{sphinxuseclass}{nbinput}
\begin{sphinxuseclass}{nblast}
{
\sphinxsetup{VerbatimColor={named}{nbsphinx-code-bg}}
\sphinxsetup{VerbatimBorderColor={named}{nbsphinx-code-border}}
\begin{sphinxVerbatim}[commandchars=\\\{\}]
\llap{\color{nbsphinxin}[2]:\,\hspace{\fboxrule}\hspace{\fboxsep}}\PYG{c+c1}{\PYGZsh{} typical function}

\PYG{k}{def} \PYG{n+nf}{function}\PYG{p}{(}\PYG{n}{x}\PYG{p}{)}\PYG{p}{:}
    \PYG{l+s+sd}{\PYGZsq{}\PYGZsq{}\PYGZsq{} function to calculate a function}
\PYG{l+s+sd}{    arguments:}
\PYG{l+s+sd}{        x ... float or integer value}
\PYG{l+s+sd}{    returns:}
\PYG{l+s+sd}{        y ... two times the integer value}
\PYG{l+s+sd}{    \PYGZsq{}\PYGZsq{}\PYGZsq{}}
    \PYG{n}{y}\PYG{o}{=}\PYG{l+m+mi}{2}\PYG{o}{*}\PYG{n}{x} \PYG{c+c1}{\PYGZsh{} don\PYGZsq{}t forget the identation of the block}
    \PYG{k}{return}\PYG{p}{(}\PYG{n}{y}\PYG{p}{)}
\end{sphinxVerbatim}
}

\end{sphinxuseclass}
\end{sphinxuseclass}
\begin{sphinxuseclass}{nbinput}
{
\sphinxsetup{VerbatimColor={named}{nbsphinx-code-bg}}
\sphinxsetup{VerbatimBorderColor={named}{nbsphinx-code-border}}
\begin{sphinxVerbatim}[commandchars=\\\{\}]
\llap{\color{nbsphinxin}[3]:\,\hspace{\fboxrule}\hspace{\fboxsep}}\PYG{n}{help}\PYG{p}{(}\PYG{n}{function}\PYG{p}{)}
\end{sphinxVerbatim}
}

\end{sphinxuseclass}
\begin{sphinxuseclass}{nboutput}
\begin{sphinxuseclass}{nblast}
{

\kern-\sphinxverbatimsmallskipamount\kern-\baselineskip
\kern+\FrameHeightAdjust\kern-\fboxrule
\vspace{\nbsphinxcodecellspacing}

\sphinxsetup{VerbatimColor={named}{white}}
\sphinxsetup{VerbatimBorderColor={named}{nbsphinx-code-border}}
\begin{sphinxuseclass}{output_area}
\begin{sphinxuseclass}{}


\begin{sphinxVerbatim}[commandchars=\\\{\}]
Help on function function in module \_\_main\_\_:

function(x)
    function to calculate a function
    arguments:
        x {\ldots} float or integer value
    returns:
        y {\ldots} two times the integer value

\end{sphinxVerbatim}



\end{sphinxuseclass}
\end{sphinxuseclass}
}

\end{sphinxuseclass}
\end{sphinxuseclass}

\chapter{Entering Markdown}
\label{\detokenize{notebooks/Intro/EditCells:Entering-Markdown}}
\sphinxAtStartPar
Text can be added to Jupyter Notebooks using Markdown cells. This is extremely useful providing a complete documentation of your calculations or simulations. In this way, everything really becomes a notebook. You can change the cell type to Markdown by using the “Cell Actions” menu, or with a hardware keyboard shortcut \sphinxcode{\sphinxupquote{m}}. Markdown is a popular markup language that is a superset of HTML. Its specification can be found here:

\sphinxAtStartPar
\sphinxurl{https://daringfireball.net/projects/markdown/}

\sphinxAtStartPar
Markdown cells can either be \sphinxstylestrong{rendered} or \sphinxstylestrong{unrendered}.

\sphinxAtStartPar
When they are rendered, you will see a nice formatted representation of the cell’s contents.

\sphinxAtStartPar
When they are unrendered, you will see the raw text source of the cell. To render the selected cell, click the \sphinxcode{\sphinxupquote{▶︎}} button or \sphinxcode{\sphinxupquote{shift}}+ \sphinxcode{\sphinxupquote{enter}}. To unrender, select the markdown cell, and press \sphinxcode{\sphinxupquote{enter}} or just double click.


\section{Markdown basics}
\label{\detokenize{notebooks/Intro/EditCells:Markdown-basics}}
\sphinxAtStartPar
Below are some basic markdown examples, in its rendered form. If you which to access how to create specific appearances, double click the individual cells to put them into an unrendered edit mode.

\sphinxAtStartPar
You can make text \sphinxstyleemphasis{italic} or \sphinxstylestrong{bold}.

\sphinxAtStartPar
You can build nested itemized or enumerated lists:
\begin{itemize}
\item {} 
\sphinxAtStartPar
first item
\begin{itemize}
\item {} 
\sphinxAtStartPar
first subitem
\begin{itemize}
\item {} 
\sphinxAtStartPar
first subsubitem

\end{itemize}

\item {} 
\sphinxAtStartPar
second subitem \sphinxhyphen{} first subitem of second subitem \sphinxhyphen{} second subitem of second subitem

\end{itemize}

\item {} 
\sphinxAtStartPar
second item
\begin{itemize}
\item {} 
\sphinxAtStartPar
first subitem

\end{itemize}

\item {} 
\sphinxAtStartPar
third item
\begin{itemize}
\item {} 
\sphinxAtStartPar
first subitem

\end{itemize}

\end{itemize}

\sphinxAtStartPar
Now another list:
\begin{enumerate}
\sphinxsetlistlabels{\arabic}{enumi}{enumii}{}{.}%
\item {} 
\sphinxAtStartPar
Here we go
\begin{enumerate}
\sphinxsetlistlabels{\arabic}{enumii}{enumiii}{}{.}%
\item {} 
\sphinxAtStartPar
Sublist

\item {} 
\sphinxAtStartPar
Sublist

\end{enumerate}

\item {} 
\sphinxAtStartPar
There we go

\item {} 
\sphinxAtStartPar
Now this

\end{enumerate}

\sphinxAtStartPar
Here is a blockquote:
\begin{quote}

\sphinxAtStartPar
Beautiful is better than ugly. Explicit is better than implicit. Simple is better than complex. Complex is better than complicated. Flat is better than nested. Sparse is better than dense. Readability counts. Special cases aren’t special enough to break the rules. Although practicality beats purity. Errors should never pass silently. Unless explicitly silenced. In the face of ambiguity, refuse the temptation to guess. There should be one\textendash{} and preferably only one \textendash{}obvious way to do it.
Although that way may not be obvious at first unless you’re Dutch. Now is better than never. Although never is often better than \sphinxstyleemphasis{right} now. If the implementation is hard to explain, it’s a bad idea. If the implementation is easy to explain, it may be a good idea. Namespaces are one honking great idea \textendash{} let’s do more of those!
\end{quote}

\sphinxAtStartPar
And shorthand for links:

\sphinxAtStartPar
\sphinxhref{http://jupyter.org}{Jupyter’s website}


\section{Headings}
\label{\detokenize{notebooks/Intro/EditCells:Headings}}
\sphinxAtStartPar
You can add headings by starting a line with one (or multiple) \sphinxcode{\sphinxupquote{\#}} followed by a space, as in the following example:

\begin{sphinxVerbatim}[commandchars=\\\{\}]
\PYGZsh{} Heading 1
\PYGZsh{} Heading 2
\PYGZsh{}\PYGZsh{} Heading 2.1
\PYGZsh{}\PYGZsh{} Heading 2.2
\end{sphinxVerbatim}


\section{Embedded code}
\label{\detokenize{notebooks/Intro/EditCells:Embedded-code}}
\sphinxAtStartPar
You can embed code meant for illustration instead of execution in Python:

\begin{sphinxVerbatim}[commandchars=\\\{\}]
def f(x):
    \PYGZdq{}\PYGZdq{}\PYGZdq{}a docstring\PYGZdq{}\PYGZdq{}\PYGZdq{}
    return x**2
\end{sphinxVerbatim}

\sphinxAtStartPar
or other languages:

\begin{sphinxVerbatim}[commandchars=\\\{\}]
if (i=0; i\PYGZlt{}n; i++) \PYGZob{}
  printf(\PYGZdq{}hello \PYGZpc{}d\PYGZbs{}n\PYGZdq{}, i);
  x += 4;
\PYGZcb{}
\end{sphinxVerbatim}


\section{LaTeX equations}
\label{\detokenize{notebooks/Intro/EditCells:LaTeX-equations}}
\sphinxAtStartPar
Courtesy of MathJax, you can include mathematical expressions both inline: \(e^{i\pi} + 1 = 0\) and displayed:
\begin{equation*}
\begin{split}e^x=\sum_{i=0}^\infty \frac{1}{i!}x^i\end{split}
\end{equation*}
\sphinxAtStartPar
Inline expressions can be added by surrounding the latex code with \sphinxcode{\sphinxupquote{\$}}:

\begin{sphinxVerbatim}[commandchars=\\\{\}]
\PYGZdl{}e\PYGZca{}\PYGZob{}i\PYGZbs{}pi\PYGZcb{} + 1 = 0\PYGZdl{}
\end{sphinxVerbatim}

\sphinxAtStartPar
Expressions on their own line are surrounded by \sphinxcode{\sphinxupquote{\$\$}}:

\begin{sphinxVerbatim}[commandchars=\\\{\}]
\PYG{l+s+sb}{\PYGZdl{}\PYGZdl{}}\PYG{n+nb}{e}\PYG{n+nb}{\PYGZca{}}\PYG{n+nb}{x}\PYG{o}{=}\PYG{n+nv}{\PYGZbs{}sum}\PYG{n+nb}{\PYGZus{}}\PYG{n+nb}{\PYGZob{}}\PYG{n+nb}{i}\PYG{o}{=}\PYG{l+m}{0}\PYG{n+nb}{\PYGZcb{}}\PYG{n+nb}{\PYGZca{}}\PYG{n+nv}{\PYGZbs{}infty}\PYG{n+nb}{ }\PYG{n+nv}{\PYGZbs{}frac}\PYG{n+nb}{\PYGZob{}}\PYG{l+m}{1}\PYG{n+nb}{\PYGZcb{}}\PYG{n+nb}{\PYGZob{}}\PYG{n+nb}{i}\PYG{o}{!}\PYG{n+nb}{\PYGZcb{}}\PYG{n+nb}{x}\PYG{n+nb}{\PYGZca{}}\PYG{n+nb}{i}\PYG{l+s}{\PYGZdl{}\PYGZdl{}}
\end{sphinxVerbatim}


\section{Images}
\label{\detokenize{notebooks/Intro/EditCells:Images}}
\sphinxAtStartPar
Images may be also directly integrated into a Markdown block.

\sphinxAtStartPar
To include images use

\begin{sphinxVerbatim}[commandchars=\\\{\}]
![alternative text](url)
\end{sphinxVerbatim}

\sphinxAtStartPar
for example

\noindent\sphinxincludegraphics[width=200\sphinxpxdimen,height=200\sphinxpxdimen]{{particle}.png}


\section{Videos}
\label{\detokenize{notebooks/Intro/EditCells:Videos}}
\sphinxAtStartPar
To include videos, we use HTML code like

\begin{sphinxVerbatim}[commandchars=\\\{\}]
\PYGZlt{}video src=\PYGZdq{}mov/movie.mp4\PYGZdq{} width=\PYGZdq{}320\PYGZdq{} height=\PYGZdq{}200\PYGZdq{} controls preload\PYGZgt{}\PYGZlt{}/video\PYGZgt{}
\end{sphinxVerbatim}

\sphinxAtStartPar
in the Markdown cell. This works with videos stored locally.





\sphinxAtStartPar
You can embed YouTube Videos as well by using the \sphinxcode{\sphinxupquote{IPython}} module.

\begin{sphinxuseclass}{nbinput}
{
\sphinxsetup{VerbatimColor={named}{nbsphinx-code-bg}}
\sphinxsetup{VerbatimBorderColor={named}{nbsphinx-code-border}}
\begin{sphinxVerbatim}[commandchars=\\\{\}]
\llap{\color{nbsphinxin}[9]:\,\hspace{\fboxrule}\hspace{\fboxsep}}\PYG{k+kn}{from} \PYG{n+nn}{IPython}\PYG{n+nn}{.}\PYG{n+nn}{display} \PYG{k+kn}{import} \PYG{n}{YouTubeVideo}
\PYG{n}{YouTubeVideo}\PYG{p}{(}\PYG{l+s+s1}{\PYGZsq{}}\PYG{l+s+s1}{QlLx32juGzI}\PYG{l+s+s1}{\PYGZsq{}}\PYG{p}{,}\PYG{n}{width}\PYG{o}{=}\PYG{l+m+mi}{600}\PYG{p}{)}
\end{sphinxVerbatim}
}

\end{sphinxuseclass}
\begin{sphinxuseclass}{nboutput}
\begin{sphinxuseclass}{nblast}
\hrule height -\fboxrule\relax
\vspace{\nbsphinxcodecellspacing}

\savebox\nbsphinxpromptbox[0pt][r]{\color{nbsphinxout}\Verb|\strut{[9]:}\,|}

\begin{nbsphinxfancyoutput}

\begin{sphinxuseclass}{output_area}
\begin{sphinxuseclass}{}
\noindent\sphinxincludegraphics[width=480\sphinxpxdimen,height=360\sphinxpxdimen]{{notebooks_Intro_EditCells_26_0}.jpg}

\end{sphinxuseclass}
\end{sphinxuseclass}
\end{nbsphinxfancyoutput}

\end{sphinxuseclass}
\end{sphinxuseclass}

\chapter{Lecture Contents}
\label{\detokenize{lectures/L1/overview_1:lecture-contents}}\label{\detokenize{lectures/L1/overview_1::doc}}
\sphinxAtStartPar
This is our first lecture, in which we will have a look at the organization of the course and start off with the basic ingredients of \sphinxstylestrong{Geometrical Optics}.
While during Experimental Physics 2, you did already address electrical oscillators and the wave equation, we would like to start with Geometrical Optics here, as this is the simplest form of describing light propagation and it turns out to be a limit of electromagnetic optics. Many of the phenomena in optics can already be sufficiently understood in this limit.

\noindent\sphinxincludegraphics[width=600\sphinxpxdimen]{{slides}.png}

\sphinxAtStartPar
Lecture 1 slides for download \sphinxcode{\sphinxupquote{pdf}}

\sphinxAtStartPar
The following section was created from \sphinxcode{\sphinxupquote{/home/lectures/exp3/source/notebooks/L1/Geometrical Optics.ipynb}}.


\chapter{Geometrical Optics}
\label{\detokenize{notebooks/L1/Geometrical Optics:Geometrical-Optics}}\label{\detokenize{notebooks/L1/Geometrical Optics::doc}}
\sphinxAtStartPar
Geometrical optics refers to a field of optics that describes light propagation in terms of rays. This is on one side borrowed from your daily experience, for example, with shadows. On a more scientific side, it will turn out that geometrical optics is the approximation in which the wavelength of light is much shorter.

\begin{sphinxadmonition}{note}{}\unskip
\sphinxAtStartPar
\sphinxstylestrong{Geometrical Optics}

\sphinxAtStartPar
Geometrical optics is an approximate description of light propagation in the limit of infinitely small wavelength, where all wave phenomena like diffraction can be neglected.
\end{sphinxadmonition}


\section{Assumptions}
\label{\detokenize{notebooks/L1/Geometrical Optics:Assumptions}}
\sphinxAtStartPar
As geometrical optics is not a rigorous description, we have to write down some postulates (things we do not understand yet) to describe light propagation in geometrical optics
\begin{itemize}
\item {} 
\sphinxAtStartPar
light rays emerge from a light source

\item {} 
\sphinxAtStartPar
light rays are detected by a detector

\item {} 
\sphinxAtStartPar
light propagates in straight line paths (rays) in a homogeneous medium

\item {} 
\sphinxAtStartPar
light\textendash{}matter interaction is characterized by a refractive index \(n\)

\item {} 
\sphinxAtStartPar
light bends to a curved path in inhomogeneous media (i.e., \(n(\vec{r})\))

\item {} 
\sphinxAtStartPar
rays may be reflected and refracted at interfaces between media

\end{itemize}

\sphinxAtStartPar
Let us have a look at some examples providing indications for the linear propagation of light.

\sphinxAtStartPar
\sphinxstylestrong{Laser}

\sphinxAtStartPar
\sphinxstylestrong{Pinhole Camera}


\begin{savenotes}\sphinxattablestart
\centering
\begin{tabulary}{\linewidth}[t]{|T|}
\hline
\sphinxstyletheadfamily 
\sphinxAtStartPar
\sphinxincludegraphics[width=0.800\linewidth]{{pinhole_camera}.png}
\\
\hline
\sphinxAtStartPar
\sphinxstylestrong{Fig.:} Schematics of a pinhole camera.
\\
\hline
\end{tabulary}
\par
\sphinxattableend\end{savenotes}


\begin{savenotes}\sphinxattablestart
\centering
\begin{tabulary}{\linewidth}[t]{|T|}
\hline
\sphinxstyletheadfamily 
\sphinxAtStartPar
\sphinxincludegraphics[width=0.400\linewidth]{{pinhole_cameraF}.png}
\\
\hline
\sphinxAtStartPar
\sphinxstylestrong{Fig.:} Image of an upright letter “F” on the screen of a pinhole camera in lecture 1.
\\
\hline
\end{tabulary}
\par
\sphinxattableend\end{savenotes}


\begin{savenotes}\sphinxattablestart
\centering
\begin{tabulary}{\linewidth}[t]{|T|}
\hline
\sphinxstyletheadfamily 
\sphinxAtStartPar
\sphinxincludegraphics[width=0.400\linewidth]{{core_shadow}.png}
\\
\hline
\sphinxAtStartPar
\sphinxstylestrong{Fig.:} Core shadow and partial shadows behind a disk when illuminated with two seperate light sources.
\\
\hline
\end{tabulary}
\par
\sphinxattableend\end{savenotes}

\sphinxAtStartPar
\sphinxstylestrong{Water Basin with Salt}


\begin{savenotes}\sphinxattablestart
\centering
\begin{tabulary}{\linewidth}[t]{|T|}
\hline
\sphinxstyletheadfamily 
\sphinxAtStartPar
\sphinxincludegraphics[width=0.400\linewidth]{{refractive_index_gradient}.png}
\\
\hline
\sphinxAtStartPar
\sphinxstylestrong{Fig.:} Bend rays in a refractive index gradient with a salt (NaCl) layer on the bottom.
\\
\hline
\end{tabulary}
\par
\sphinxattableend\end{savenotes}



\sphinxAtStartPar
The following section was created from \sphinxcode{\sphinxupquote{/home/lectures/exp3/source/notebooks/L1/Reflection.ipynb}}.


\chapter{Reflection}
\label{\detokenize{notebooks/L1/Reflection:Reflection}}\label{\detokenize{notebooks/L1/Reflection::doc}}
\sphinxAtStartPar
The law of reflection is probably the most simple one. Yet the simplicity gives us the chance to define some basic objects which we will further use for the description of light rays and their propagation.


\section{Law of Reflection}
\label{\detokenize{notebooks/L1/Reflection:Law-of-Reflection}}
\sphinxAtStartPar
The sketch below shows the reflection of an incoming light ray (red) on an interface. This incoming light ray has an angle \(\theta_{1}\) with the axis (dashed line), which is perpendicular to the reflecting surface. As compared to X\sphinxhyphen{}ray diffraction, we measure the angle to the normal of the surface and not to the surface itself.


\begin{savenotes}\sphinxattablestart
\centering
\begin{tabulary}{\linewidth}[t]{|T|}
\hline
\sphinxstyletheadfamily 
\sphinxAtStartPar
\sphinxincludegraphics[width=0.490\linewidth]{{reflection}.png} \sphinxincludegraphics[width=0.500\linewidth]{{reflection_law}.png}
\\
\hline
\sphinxAtStartPar
\sphinxstylestrong{Fig.:} Law of reflection
\\
\hline
\end{tabulary}
\par
\sphinxattableend\end{savenotes}

\sphinxAtStartPar
The law of reflection tells us now, that the outgoing reflected ray is now leaving the surface under an angle \(\theta_2=\theta_1\). So both angles are the same for the reflection.

\begin{sphinxadmonition}{note}{}\unskip
\sphinxAtStartPar
\sphinxstylestrong{Law of Reflection}

\sphinxAtStartPar
If a ray is incident to a reflecting surface under an angle \(\theta_1\) it will be reflected towards under an angle \(\theta_2=\theta_1\) to the same side of the surface.
\end{sphinxadmonition}


\section{Fermat’s Principle}
\label{\detokenize{notebooks/L1/Reflection:Fermat_u2019s-Principle}}
\sphinxAtStartPar
The law of reflection can be actually obtained from a variational principle saying the light rays propagate along those path on which the propagation time is an extremum. This variational principle is called Fermat’s principle.


\begin{savenotes}\sphinxattablestart
\centering
\begin{tabulary}{\linewidth}[t]{|T|}
\hline
\sphinxstyletheadfamily 
\sphinxAtStartPar
\sphinxincludegraphics[width=0.600\linewidth]{{fermat}.png}
\\
\hline
\sphinxAtStartPar
\sphinxstylestrong{Fig.:} Sketch for deriving the law of reflection from Fermat’s principle
\\
\hline
\end{tabulary}
\par
\sphinxattableend\end{savenotes}

\sphinxAtStartPar
Consider now a light ray that should travel from point \(A\) to point \(C\) via a point \(B\) on the mirror surface. In general multiple paths are possible such as the one indicated in the above picture. Clearly this path is not satisfying our reflection law formulated above. Fermat’s principle now restricts the path length from \(A\) to \(C\) to be the one, which takes the least amount of time.

\begin{sphinxadmonition}{note}{}\unskip
\sphinxAtStartPar
\sphinxstylestrong{Fermat’s principle}

\sphinxAtStartPar
The path taken by a ray between two given points A, B is the path that can be traversed in the least time.

\sphinxAtStartPar
\sphinxstyleemphasis{More precise alternative:} A ray going in a certain particular path has the property that if we make a small change in the ray in any manner whatever, say in the location at which it comes to the mirror, or the shape of the curve, or anything, there will be no first\sphinxhyphen{}order change in the time; there will be only a second\sphinxhyphen{}order change in the time.
\end{sphinxadmonition}

\sphinxAtStartPar
So let us consider that contraints to the path length. The total length the light hast to travel via the three points is

\sphinxAtStartPar
\begin{equation}
l=l_{1}+l_{2}=\sqrt{(x-x_1)^2+y_1^2}+\sqrt{(x_2-x)^2+y_2^2}.
\end{equation}

\sphinxAtStartPar
The time that is required by the light to travel that distance \(l\) is then given by

\sphinxAtStartPar
\begin{equation}
t=\frac{l}{c},
\end{equation}

\sphinxAtStartPar
where \(c\) is the speed of light in the medium above the mirror. If this time should now be a minimum, we have to take the derivative of the time \(t\) with respect to the position \(x\) on the mirror and set that to zero, i.e.,
\begin{equation*}
\begin{split}\frac{\mathrm dt}{\mathrm dx}=0.\end{split}
\end{equation*}
\sphinxAtStartPar
This results in \begin{equation}
\frac{x-x_1}{\sqrt{(x-x_1)^2+y_{1}^2}}=\frac{x_2-x}{\sqrt{(x_2-x)^2+y_{2}^2}},
\end{equation}

\sphinxAtStartPar
which is actually

\sphinxAtStartPar
\begin{equation}
\frac{x-x_1}{l_1}=\frac{x_2-x}{l_2}
\end{equation}

\sphinxAtStartPar
or

\sphinxAtStartPar
\begin{equation}
\sin(\theta_1)=\sin(\theta_2)
\end{equation}

\sphinxAtStartPar
which finally requires
\begin{equation*}
\begin{split}\theta_1=\theta_2\end{split}
\end{equation*}
\sphinxAtStartPar
and is our law of reflection. Thus, refraction (of course) satisfies Fermat’s principle.

\sphinxAtStartPar
While this has been a special example of how to apply Fermat’s principle we can define a more general version of it correponding to the following situation also involving an inhomogeneous refractive index \(n(\vec{r})\).


\begin{savenotes}\sphinxattablestart
\centering
\begin{tabulary}{\linewidth}[t]{|T|}
\hline
\sphinxstyletheadfamily 
\sphinxAtStartPar
\sphinxincludegraphics[width=0.400\linewidth]{{fermat_general}.png}
\\
\hline
\sphinxAtStartPar
\sphinxstylestrong{Fig.:} Sketch for a general description of Fermat’s principle.
\\
\hline
\end{tabulary}
\par
\sphinxattableend\end{savenotes}

\sphinxAtStartPar
For this general scenario of light traveling along a path, Fermat’s principle is given by

\sphinxAtStartPar
\begin{equation}
\delta \int\limits_{A}^{C} n(\vec{r}) \mathrm ds=0,
\end{equation}

\sphinxAtStartPar
where the \(\delta\) denotes a variation of the path indicated.

\sphinxAtStartPar
The following section was created from \sphinxcode{\sphinxupquote{/home/lectures/exp3/source/notebooks/L1/Refraction.ipynb}}.


\chapter{Refraction}
\label{\detokenize{notebooks/L1/Refraction:Refraction}}\label{\detokenize{notebooks/L1/Refraction::doc}}
\sphinxAtStartPar
The law of refraction is the second important law of geometrical optics. It connects the refractive index \(n_1\) at the incident side and the angle of incidence \(\theta_1\) to the refractive index \(n_2\) and the angle of refraction \(\theta_2\) on the transmission side. It will later turn out, that both laws, the law of reflection and the law of refraction actually correspond to a conservation of momentum.


\section{Refractive Index}
\label{\detokenize{notebooks/L1/Refraction:Refractive-Index}}
\sphinxAtStartPar
The refractive index \(n\) is a material constant representing the factor by which the speed of light is slowed down in the medium. As the maximum speed is the speed of light in vacuum, the refractive index is typically a number \(n \ge 1\). Yet, it will turn out later, that the refractive index can be smaller than 1 or even negative. This, however, requires first to understand the origin of the refractive index.


\section{Snells Law}
\label{\detokenize{notebooks/L1/Refraction:Snells-Law}}

\begin{savenotes}\sphinxattablestart
\centering
\begin{tabulary}{\linewidth}[t]{|T|}
\hline
\sphinxstyletheadfamily 
\sphinxAtStartPar
\sphinxincludegraphics[width=0.590\linewidth]{{snell}.png} \sphinxincludegraphics[width=0.400\linewidth]{{refraction_law}.png}
\\
\hline
\sphinxAtStartPar
\sphinxstylestrong{Fig.:} Snell’s law.
\\
\hline
\end{tabulary}
\par
\sphinxattableend\end{savenotes}

\begin{sphinxadmonition}{note}{}\unskip
\sphinxAtStartPar
\sphinxstylestrong{Law of Reflection (Snells Law)}

\sphinxAtStartPar
The law of refraction (Snell’s law) is given for the above sketch by

\sphinxAtStartPar
\begin{equation}
n_1 \sin(\theta_1)=n_2 \sin(\theta_2)
\end{equation}
\end{sphinxadmonition}


\bigskip\hrule\bigskip


\sphinxAtStartPar
In the previous section in refraction we have derived the law of reflection from Fermat’s principle. \sphinxstylestrong{Try yourself to obtain Snell’s law with the help of Fermat’s principle}. Check out the \sphinxhref{https://mybinder.org/v2/gh/fcichos/exp3/master?urlpath=tree/source/snippets/Refraction\%20Explorer.ipynb}{Refraction Explorer Snippet} if you want to play around with Snells law. You may tray there even negative refraction.


\bigskip\hrule\bigskip


\sphinxAtStartPar
According to Snell’s law, there is a general behavior of the corresponding angles, which you might want to remember (see also Fig. above). Consider the following cases:

\sphinxAtStartPar
\(n_1<n_2\): \sphinxhyphen{} refraction is towards the optical axis \sphinxhyphen{} \(\theta_2<\theta_1\)

\sphinxAtStartPar
\(n_2<n_1\): \sphinxhyphen{} refraction is away from the optical axis \sphinxhyphen{} \(\theta_1<\theta_2\)

\sphinxAtStartPar
The plot below shows this result in three plots with varying incident angle and two different refractive index combinations (glass/air, ait/glass).


\begin{savenotes}\sphinxattablestart
\centering
\begin{tabulary}{\linewidth}[t]{|T|}
\hline
\sphinxstyletheadfamily 
\sphinxAtStartPar
\sphinxincludegraphics[width=0.600\linewidth]{{angles}.png}
\\
\hline
\sphinxAtStartPar
\sphinxstylestrong{Fig.:} Snells law for different combinations of refractive indices.
\\
\hline
\end{tabulary}
\par
\sphinxattableend\end{savenotes}


\section{Total Internal Reflection}
\label{\detokenize{notebooks/L1/Refraction:Total-Internal-Reflection}}
\sphinxAtStartPar
The above diagram reveals a special case occurring when \(n_1>n_2\). Under these condition, one may change the incident angle \(\theta_1\) such that the outgoing angle becomes \(\theta_2=\frac{\pi}{2}\). For any incident angle \(\theta_1\) larger than this critical angle, the is no refracted ray anymore, but just a reflected ray. This is despite the fact that the material with \(n_2\) is completely transparent. This phenomenon is called \sphinxstylestrong{total internal reflection} and it
has several important applications.

\sphinxAtStartPar
Let’s first formalize this. Using the Snell’s law. For \(\theta_2=\frac{\pi}{2}\) we obtain
\begin{equation*}
\begin{split}\theta_1=\theta_c=\sin^{-1}\left (\frac{n_2}{n_1}\right )\end{split}
\end{equation*}
\sphinxAtStartPar
for the critical angle \(\theta_c\). As the \(\sin^{-1}()\) requires an argument \(\le1\), this works only if we have \(n_2 < n_1\)


\begin{savenotes}\sphinxattablestart
\centering
\begin{tabulary}{\linewidth}[t]{|T|}
\hline
\sphinxstyletheadfamily 
\sphinxAtStartPar
\sphinxincludegraphics[width=0.500\linewidth]{{tir}.png} \sphinxincludegraphics[width=0.490\linewidth]{{tir_disk}.png}
\\
\hline
\sphinxAtStartPar
\sphinxstylestrong{Fig.:} Total internal reflection
\\
\hline
\end{tabulary}
\par
\sphinxattableend\end{savenotes}

\begin{sphinxadmonition}{note}{}\unskip
\sphinxAtStartPar
\sphinxstylestrong{Total Internal Reflection}

\sphinxAtStartPar
Total internal reflection occurs when light is passing from higher refractive index to lower refractive index materials for incidence angle larger than a critical angle

\sphinxAtStartPar
\begin{equation}
\theta_c=\sin^{-1}\left (\frac{n_2}{n_1}\right )
\end{equation}
\end{sphinxadmonition}

\sphinxAtStartPar
We can demonstrate total internal reflection very easily with a water basin, for example, where we couple in light from a laser from the side.


\begin{savenotes}\sphinxattablestart
\centering
\begin{tabulary}{\linewidth}[t]{|T|}
\hline
\sphinxstyletheadfamily 
\sphinxAtStartPar
\sphinxincludegraphics[width=0.590\linewidth]{{basin_tir}.png} \sphinxincludegraphics[width=0.400\linewidth]{{tir_basin}.png}
\\
\hline
\sphinxAtStartPar
\sphinxstylestrong{Fig.:} Total internal reflection at a water/air interface.
\\
\hline
\end{tabulary}
\par
\sphinxattableend\end{savenotes}

\sphinxAtStartPar
But you could try that yourself also in the bath tub diving below the water surface.

\sphinxAtStartPar
\sphinxstylestrong{Optical Fiber} Total internal reflection is very important for guiding light in telecommunications, for example. There, glass wires with a diameter from a few to several 100 µm are used to transport light. The glass wire with a central core of refractive index \(n_1\) is surrounded by a cladding layer of slightly lower refractive index \(n_2\). Light is then coupled into the fiber from one side. To obtain total internal reflection in this setting, the incident rays have to hit the
front of the fiber at a maximum angle \(\theta_{a}\)


\begin{savenotes}\sphinxattablestart
\centering
\begin{tabulary}{\linewidth}[t]{|T|}
\hline
\sphinxstyletheadfamily 
\sphinxAtStartPar
\sphinxincludegraphics[width=0.590\linewidth]{{fiber}.png} \sphinxincludegraphics[width=0.400\linewidth]{{tir_rod}.png}
\\
\hline
\sphinxAtStartPar
\sphinxstylestrong{Fig.:} Total internal reflection in an optical fiber and a glass rod.
\\
\hline
\end{tabulary}
\par
\sphinxattableend\end{savenotes}

\sphinxAtStartPar
The angle \(\theta_{a}\) can be easily calculated from Snells law. To characterize this opening angle one typically defines a new quantity called numerical aperature \(NA\), which is the sine of the opening angle \(\theta_a\)

\sphinxAtStartPar
\begin{equation}
NA=\sin(\theta_a)=\sqrt{n_1^2-n_2^{2}}
\end{equation}

\sphinxAtStartPar
Using typical values of the refractive indices \(n_1=1.475\) and \(n_2=1.46\) one obtains a numerical apeture of \(NA\approx 0.2\).

\sphinxAtStartPar
The following section was created from \sphinxcode{\sphinxupquote{/home/lectures/exp3/source/snippets/Refraction Explorer.ipynb}}.


\chapter{EXP3 Snippets \textendash{} Refraction Explorer}
\label{\detokenize{snippets/Refraction Explorer:EXP3-Snippets-_-Refraction-Explorer}}\label{\detokenize{snippets/Refraction Explorer::doc}}
\sphinxAtStartPar
This is a small python code snippet, which you can explore on the \sphinxstyleemphasis{myBinder} service with the button on the top of this webpage. It shows the refraction of a light ray (red) incident to an interface (horizontal line), which is then refracted. The interface is seperating two areas with different refractive index \(n_1, n_2\), which you can modify with the sliders in the same way as the incident angle. The refractive index \(n_2\) may even go negative and you may want to explore what
happens then.
\begin{enumerate}
\sphinxsetlistlabels{\alph}{enumi}{enumii}{(}{)}%
\setcounter{enumi}{2}
\item {} \begin{enumerate}
\sphinxsetlistlabels{\Alph}{enumii}{enumiii}{}{.}%
\setcounter{enumii}{5}
\item {} 
\sphinxAtStartPar
Cichos 2020

\end{enumerate}

\end{enumerate}

\begin{sphinxuseclass}{nbinput}
\begin{sphinxuseclass}{nblast}
{
\sphinxsetup{VerbatimColor={named}{nbsphinx-code-bg}}
\sphinxsetup{VerbatimBorderColor={named}{nbsphinx-code-border}}
\begin{sphinxVerbatim}[commandchars=\\\{\}]
\llap{\color{nbsphinxin}[1]:\,\hspace{\fboxrule}\hspace{\fboxsep}}\PYG{o}{\PYGZpc{}}\PYG{k}{matplotlib} widget
\PYG{k+kn}{import} \PYG{n+nn}{ipywidgets} \PYG{k}{as} \PYG{n+nn}{widgets}
\PYG{k+kn}{import} \PYG{n+nn}{matplotlib}\PYG{n+nn}{.}\PYG{n+nn}{pyplot} \PYG{k}{as} \PYG{n+nn}{plt}
\PYG{k+kn}{import} \PYG{n+nn}{numpy} \PYG{k}{as} \PYG{n+nn}{np}
\end{sphinxVerbatim}
}

\end{sphinxuseclass}
\end{sphinxuseclass}
\begin{sphinxuseclass}{nbinput}
\begin{sphinxuseclass}{nblast}
{
\sphinxsetup{VerbatimColor={named}{nbsphinx-code-bg}}
\sphinxsetup{VerbatimBorderColor={named}{nbsphinx-code-border}}
\begin{sphinxVerbatim}[commandchars=\\\{\}]
\llap{\color{nbsphinxin}[2]:\,\hspace{\fboxrule}\hspace{\fboxsep}}\PYG{k}{def} \PYG{n+nf}{magnitude}\PYG{p}{(}\PYG{n}{vector}\PYG{p}{)}\PYG{p}{:}
   \PYG{k}{return} \PYG{n}{np}\PYG{o}{.}\PYG{n}{sqrt}\PYG{p}{(}\PYG{n}{np}\PYG{o}{.}\PYG{n}{dot}\PYG{p}{(}\PYG{n}{np}\PYG{o}{.}\PYG{n}{array}\PYG{p}{(}\PYG{n}{vector}\PYG{p}{)}\PYG{p}{,}\PYG{n}{np}\PYG{o}{.}\PYG{n}{array}\PYG{p}{(}\PYG{n}{vector}\PYG{p}{)}\PYG{p}{)}\PYG{p}{)}

\PYG{k}{def} \PYG{n+nf}{norm}\PYG{p}{(}\PYG{n}{vector}\PYG{p}{)}\PYG{p}{:}
   \PYG{k}{return} \PYG{n}{np}\PYG{o}{.}\PYG{n}{array}\PYG{p}{(}\PYG{n}{vector}\PYG{p}{)}\PYG{o}{/}\PYG{n}{magnitude}\PYG{p}{(}\PYG{n}{np}\PYG{o}{.}\PYG{n}{array}\PYG{p}{(}\PYG{n}{vector}\PYG{p}{)}\PYG{p}{)}

\PYG{k}{def} \PYG{n+nf}{lineRayIntersectionPoint}\PYG{p}{(}\PYG{n}{rayOrigin}\PYG{p}{,} \PYG{n}{rayDirection}\PYG{p}{,} \PYG{n}{point1}\PYG{p}{,} \PYG{n}{point2}\PYG{p}{)}\PYG{p}{:}
        \PYG{c+c1}{\PYGZsh{} Convert to numpy arrays}
    \PYG{n}{rayOrigin} \PYG{o}{=} \PYG{n}{np}\PYG{o}{.}\PYG{n}{array}\PYG{p}{(}\PYG{n}{rayOrigin}\PYG{p}{,} \PYG{n}{dtype}\PYG{o}{=}\PYG{n}{np}\PYG{o}{.}\PYG{n}{float}\PYG{p}{)}
    \PYG{n}{rayDirection} \PYG{o}{=} \PYG{n}{np}\PYG{o}{.}\PYG{n}{array}\PYG{p}{(}\PYG{n}{norm}\PYG{p}{(}\PYG{n}{rayDirection}\PYG{p}{)}\PYG{p}{,} \PYG{n}{dtype}\PYG{o}{=}\PYG{n}{np}\PYG{o}{.}\PYG{n}{float}\PYG{p}{)}
    \PYG{n}{point1} \PYG{o}{=} \PYG{n}{np}\PYG{o}{.}\PYG{n}{array}\PYG{p}{(}\PYG{n}{point1}\PYG{p}{,} \PYG{n}{dtype}\PYG{o}{=}\PYG{n}{np}\PYG{o}{.}\PYG{n}{float}\PYG{p}{)}
    \PYG{n}{point2} \PYG{o}{=} \PYG{n}{np}\PYG{o}{.}\PYG{n}{array}\PYG{p}{(}\PYG{n}{point2}\PYG{p}{,} \PYG{n}{dtype}\PYG{o}{=}\PYG{n}{np}\PYG{o}{.}\PYG{n}{float}\PYG{p}{)}

    \PYG{c+c1}{\PYGZsh{} Ray\PYGZhy{}Line Segment Intersection Test in 2D}
    \PYG{c+c1}{\PYGZsh{} http://bit.ly/1CoxdrG}
    \PYG{n}{v1} \PYG{o}{=} \PYG{n}{rayOrigin} \PYG{o}{\PYGZhy{}} \PYG{n}{point1}
    \PYG{n}{v2} \PYG{o}{=} \PYG{n}{point2} \PYG{o}{\PYGZhy{}} \PYG{n}{point1}
    \PYG{n}{v3} \PYG{o}{=} \PYG{n}{np}\PYG{o}{.}\PYG{n}{array}\PYG{p}{(}\PYG{p}{[}\PYG{o}{\PYGZhy{}}\PYG{n}{rayDirection}\PYG{p}{[}\PYG{l+m+mi}{1}\PYG{p}{]}\PYG{p}{,} \PYG{n}{rayDirection}\PYG{p}{[}\PYG{l+m+mi}{0}\PYG{p}{]}\PYG{p}{]}\PYG{p}{)}
    \PYG{n}{t1} \PYG{o}{=} \PYG{n}{np}\PYG{o}{.}\PYG{n}{cross}\PYG{p}{(}\PYG{n}{v2}\PYG{p}{,} \PYG{n}{v1}\PYG{p}{)} \PYG{o}{/} \PYG{n}{np}\PYG{o}{.}\PYG{n}{dot}\PYG{p}{(}\PYG{n}{v2}\PYG{p}{,} \PYG{n}{v3}\PYG{p}{)}
    \PYG{n}{t2} \PYG{o}{=} \PYG{n}{np}\PYG{o}{.}\PYG{n}{dot}\PYG{p}{(}\PYG{n}{v1}\PYG{p}{,} \PYG{n}{v3}\PYG{p}{)} \PYG{o}{/} \PYG{n}{np}\PYG{o}{.}\PYG{n}{dot}\PYG{p}{(}\PYG{n}{v2}\PYG{p}{,} \PYG{n}{v3}\PYG{p}{)}
    \PYG{k}{if} \PYG{n}{t1} \PYG{o}{\PYGZgt{}}\PYG{o}{=} \PYG{l+m+mf}{0.0} \PYG{o+ow}{and} \PYG{n}{t2} \PYG{o}{\PYGZgt{}}\PYG{o}{=} \PYG{l+m+mf}{0.0} \PYG{o+ow}{and} \PYG{n}{t2} \PYG{o}{\PYGZlt{}}\PYG{o}{=} \PYG{l+m+mf}{1.0}\PYG{p}{:}
        \PYG{k}{return} \PYG{p}{[}\PYG{n}{rayOrigin} \PYG{o}{+} \PYG{n}{t1} \PYG{o}{*} \PYG{n}{rayDirection}\PYG{p}{]}
    \PYG{k}{return} \PYG{p}{[}\PYG{p}{]}


\end{sphinxVerbatim}
}

\end{sphinxuseclass}
\end{sphinxuseclass}
\begin{sphinxuseclass}{nbinput}
{
\sphinxsetup{VerbatimColor={named}{nbsphinx-code-bg}}
\sphinxsetup{VerbatimBorderColor={named}{nbsphinx-code-border}}
\begin{sphinxVerbatim}[commandchars=\\\{\}]
\llap{\color{nbsphinxin}[3]:\,\hspace{\fboxrule}\hspace{\fboxsep}}\PYG{n}{fig}\PYG{p}{,} \PYG{n}{ax} \PYG{o}{=} \PYG{n}{plt}\PYG{o}{.}\PYG{n}{subplots}\PYG{p}{(}\PYG{n}{figsize}\PYG{o}{=}\PYG{p}{(}\PYG{l+m+mi}{4}\PYG{p}{,} \PYG{l+m+mi}{4}\PYG{p}{)}\PYG{p}{)}
\PYG{n}{fig}\PYG{o}{.}\PYG{n}{canvas}\PYG{o}{.}\PYG{n}{header\PYGZus{}visible} \PYG{o}{=} \PYG{k+kc}{False}

\PYG{n+nd}{@widgets}\PYG{o}{.}\PYG{n}{interact}\PYG{p}{(}\PYG{n}{n1}\PYG{o}{=}\PYG{p}{(}\PYG{l+m+mi}{1}\PYG{p}{,}\PYG{l+m+mi}{2}\PYG{p}{,}\PYG{l+m+mf}{0.01}\PYG{p}{)}\PYG{p}{,}\PYG{n}{n2}\PYG{o}{=}\PYG{p}{(}\PYG{o}{\PYGZhy{}}\PYG{l+m+mi}{2}\PYG{p}{,}\PYG{l+m+mi}{3}\PYG{p}{,}\PYG{l+m+mf}{0.01}\PYG{p}{)}\PYG{p}{,} \PYG{n}{phi}\PYG{o}{=}\PYG{p}{(}\PYG{l+m+mi}{0}\PYG{p}{,} \PYG{l+m+mi}{90}\PYG{p}{,} \PYG{l+m+mf}{0.1}\PYG{p}{)}\PYG{p}{)}
\PYG{k}{def} \PYG{n+nf}{update}\PYG{p}{(}\PYG{n}{n1}\PYG{o}{=}\PYG{l+m+mi}{1}\PYG{p}{,}\PYG{n}{n2}\PYG{o}{=}\PYG{l+m+mf}{1.5}\PYG{p}{,}\PYG{n}{phi}\PYG{o}{=}\PYG{l+m+mi}{45}\PYG{p}{)}\PYG{p}{:}
    \PYG{l+s+sd}{\PYGZdq{}\PYGZdq{}\PYGZdq{}Remove old lines from plot and plot new one\PYGZdq{}\PYGZdq{}\PYGZdq{}}
    \PYG{n}{ax}\PYG{o}{.}\PYG{n}{cla}\PYG{p}{(}\PYG{p}{)}
    \PYG{n}{theta1}\PYG{o}{=}\PYG{n}{phi}\PYG{o}{*}\PYG{n}{np}\PYG{o}{.}\PYG{n}{pi}\PYG{o}{/}\PYG{l+m+mi}{180}
    \PYG{k}{if} \PYG{n}{n1}\PYG{o}{*}\PYG{n}{np}\PYG{o}{.}\PYG{n}{sin}\PYG{p}{(}\PYG{n}{theta1}\PYG{p}{)}\PYG{o}{/}\PYG{n}{n2}\PYG{o}{\PYGZlt{}}\PYG{o}{=}\PYG{l+m+mi}{1}\PYG{p}{:}
        \PYG{n}{theta2}\PYG{o}{=}\PYG{n}{np}\PYG{o}{.}\PYG{n}{arcsin}\PYG{p}{(}\PYG{n}{n1}\PYG{o}{*}\PYG{n}{np}\PYG{o}{.}\PYG{n}{sin}\PYG{p}{(}\PYG{n}{theta1}\PYG{p}{)}\PYG{o}{/}\PYG{n}{n2}\PYG{p}{)}
    \PYG{k}{else}\PYG{p}{:}
        \PYG{n}{theta2}\PYG{o}{=}\PYG{o}{\PYGZhy{}}\PYG{n}{theta1}\PYG{o}{+}\PYG{n}{np}\PYG{o}{.}\PYG{n}{pi}
    \PYG{n}{ax}\PYG{o}{.}\PYG{n}{set\PYGZus{}title}\PYG{p}{(}\PYG{l+s+s2}{\PYGZdq{}}\PYG{l+s+s2}{Refraction Explorer}\PYG{l+s+s2}{\PYGZdq{}}\PYG{p}{)}
    \PYG{n}{ax}\PYG{o}{.}\PYG{n}{axvline}\PYG{p}{(}\PYG{n}{x}\PYG{o}{=}\PYG{l+m+mi}{0}\PYG{p}{,}\PYG{n}{ls}\PYG{o}{=}\PYG{l+s+s1}{\PYGZsq{}}\PYG{l+s+s1}{\PYGZhy{}\PYGZhy{}}\PYG{l+s+s1}{\PYGZsq{}}\PYG{p}{)}
    \PYG{n}{ax}\PYG{o}{.}\PYG{n}{text}\PYG{p}{(}\PYG{o}{\PYGZhy{}}\PYG{l+m+mf}{0.04}\PYG{p}{,}\PYG{l+m+mf}{0.04}\PYG{p}{,}\PYG{l+s+sa}{r}\PYG{l+s+s1}{\PYGZsq{}}\PYG{l+s+s1}{\PYGZdl{}n\PYGZus{}2\PYGZdl{}=}\PYG{l+s+si}{\PYGZob{}\PYGZcb{}}\PYG{l+s+s1}{\PYGZsq{}}\PYG{o}{.}\PYG{n}{format}\PYG{p}{(}\PYG{n}{n2}\PYG{p}{)}\PYG{p}{)}
    \PYG{n}{ax}\PYG{o}{.}\PYG{n}{text}\PYG{p}{(}\PYG{o}{\PYGZhy{}}\PYG{l+m+mf}{0.04}\PYG{p}{,}\PYG{o}{\PYGZhy{}}\PYG{l+m+mf}{0.04}\PYG{p}{,}\PYG{l+s+sa}{r}\PYG{l+s+s1}{\PYGZsq{}}\PYG{l+s+s1}{\PYGZdl{}n\PYGZus{}1\PYGZdl{}=}\PYG{l+s+si}{\PYGZob{}\PYGZcb{}}\PYG{l+s+s1}{\PYGZsq{}}\PYG{o}{.}\PYG{n}{format}\PYG{p}{(}\PYG{n}{n1}\PYG{p}{)}\PYG{p}{)}
    \PYG{n}{ax}\PYG{o}{.}\PYG{n}{text}\PYG{p}{(}\PYG{l+m+mf}{0.03}\PYG{p}{,}\PYG{l+m+mf}{0.04}\PYG{p}{,}\PYG{l+s+sa}{r}\PYG{l+s+s1}{\PYGZsq{}}\PYG{l+s+s1}{\PYGZdl{}}\PYG{l+s+s1}{\PYGZbs{}}\PYG{l+s+s1}{theta\PYGZus{}2\PYGZdl{}=}\PYG{l+s+si}{\PYGZob{}\PYGZcb{}}\PYG{l+s+s1}{\PYGZsq{}}\PYG{o}{.}\PYG{n}{format}\PYG{p}{(}\PYG{n+nb}{round}\PYG{p}{(}\PYG{n}{theta2}\PYG{o}{*}\PYG{l+m+mi}{180}\PYG{o}{/}\PYG{n}{np}\PYG{o}{.}\PYG{n}{pi}\PYG{p}{)}\PYG{p}{,}\PYG{l+m+mi}{1}\PYG{p}{)}\PYG{p}{)}
    \PYG{n}{ax}\PYG{o}{.}\PYG{n}{text}\PYG{p}{(}\PYG{l+m+mf}{0.03}\PYG{p}{,}\PYG{o}{\PYGZhy{}}\PYG{l+m+mf}{0.04}\PYG{p}{,}\PYG{l+s+sa}{r}\PYG{l+s+s1}{\PYGZsq{}}\PYG{l+s+s1}{\PYGZdl{}}\PYG{l+s+s1}{\PYGZbs{}}\PYG{l+s+s1}{theta\PYGZus{}1\PYGZdl{}=}\PYG{l+s+si}{\PYGZob{}\PYGZcb{}}\PYG{l+s+s1}{\PYGZsq{}}\PYG{o}{.}\PYG{n}{format}\PYG{p}{(}\PYG{n+nb}{round}\PYG{p}{(}\PYG{n}{theta1}\PYG{o}{*}\PYG{l+m+mi}{180}\PYG{o}{/}\PYG{n}{np}\PYG{o}{.}\PYG{n}{pi}\PYG{p}{)}\PYG{p}{,}\PYG{l+m+mi}{1}\PYG{p}{)}\PYG{p}{)}


    \PYG{n}{ax}\PYG{o}{.}\PYG{n}{axhline}\PYG{p}{(}\PYG{n}{y}\PYG{o}{=}\PYG{l+m+mi}{0}\PYG{p}{,}\PYG{n}{color}\PYG{o}{=}\PYG{l+s+s1}{\PYGZsq{}}\PYG{l+s+s1}{k}\PYG{l+s+s1}{\PYGZsq{}}\PYG{p}{,}\PYG{n}{lw}\PYG{o}{=}\PYG{l+m+mf}{0.5}\PYG{p}{)}
    \PYG{n}{ax}\PYG{o}{.}\PYG{n}{quiver}\PYG{p}{(}\PYG{l+m+mi}{0}\PYG{p}{,}\PYG{l+m+mi}{0}\PYG{p}{,}\PYG{n}{np}\PYG{o}{.}\PYG{n}{sin}\PYG{p}{(}\PYG{n}{theta1}\PYG{p}{)}\PYG{p}{,}\PYG{n}{np}\PYG{o}{.}\PYG{n}{cos}\PYG{p}{(}\PYG{n}{theta1}\PYG{p}{)}\PYG{p}{,}\PYG{n}{scale}\PYG{o}{=}\PYG{l+m+mi}{3}\PYG{p}{,}\PYG{n}{pivot}\PYG{o}{=}\PYG{l+s+s1}{\PYGZsq{}}\PYG{l+s+s1}{tip}\PYG{l+s+s1}{\PYGZsq{}} \PYG{p}{,}\PYG{n}{color}\PYG{o}{=}\PYG{l+s+s1}{\PYGZsq{}}\PYG{l+s+s1}{red}\PYG{l+s+s1}{\PYGZsq{}}\PYG{p}{)}
    \PYG{n}{ax}\PYG{o}{.}\PYG{n}{quiver}\PYG{p}{(}\PYG{l+m+mi}{0}\PYG{p}{,}\PYG{l+m+mi}{0}\PYG{p}{,}\PYG{n}{np}\PYG{o}{.}\PYG{n}{sin}\PYG{p}{(}\PYG{n}{theta2}\PYG{p}{)}\PYG{p}{,}\PYG{n}{np}\PYG{o}{.}\PYG{n}{cos}\PYG{p}{(}\PYG{n}{theta2}\PYG{p}{)}\PYG{p}{,}\PYG{n}{scale}\PYG{o}{=}\PYG{l+m+mi}{3}\PYG{p}{,}\PYG{n}{color}\PYG{o}{=}\PYG{l+s+s1}{\PYGZsq{}}\PYG{l+s+s1}{blue}\PYG{l+s+s1}{\PYGZsq{}}\PYG{p}{)}

\end{sphinxVerbatim}
}

\end{sphinxuseclass}
\begin{sphinxuseclass}{nboutput}
{

\kern-\sphinxverbatimsmallskipamount\kern-\baselineskip
\kern+\FrameHeightAdjust\kern-\fboxrule
\vspace{\nbsphinxcodecellspacing}

\sphinxsetup{VerbatimColor={named}{white}}
\sphinxsetup{VerbatimBorderColor={named}{nbsphinx-code-border}}
\begin{sphinxuseclass}{output_area}
\begin{sphinxuseclass}{}


\begin{sphinxVerbatim}[commandchars=\\\{\}]
Canvas(toolbar=Toolbar(toolitems=[('Home', 'Reset original view', 'home', 'home'), ('Back', 'Back to previous …
\end{sphinxVerbatim}



\end{sphinxuseclass}
\end{sphinxuseclass}
}

\end{sphinxuseclass}
\begin{sphinxuseclass}{nboutput}
\begin{sphinxuseclass}{nblast}
{

\kern-\sphinxverbatimsmallskipamount\kern-\baselineskip
\kern+\FrameHeightAdjust\kern-\fboxrule
\vspace{\nbsphinxcodecellspacing}

\sphinxsetup{VerbatimColor={named}{white}}
\sphinxsetup{VerbatimBorderColor={named}{nbsphinx-code-border}}
\begin{sphinxuseclass}{output_area}
\begin{sphinxuseclass}{}


\begin{sphinxVerbatim}[commandchars=\\\{\}]
interactive(children=(FloatSlider(value=1.0, description='n1', max=2.0, min=1.0, step=0.01), FloatSlider(value…
\end{sphinxVerbatim}



\end{sphinxuseclass}
\end{sphinxuseclass}
}

\end{sphinxuseclass}
\end{sphinxuseclass}
\noindent\sphinxincludegraphics[width=400\sphinxpxdimen,height=400\sphinxpxdimen]{{refraction_explorer}.png}

\begin{sphinxuseclass}{nbinput}
\begin{sphinxuseclass}{nblast}
{
\sphinxsetup{VerbatimColor={named}{nbsphinx-code-bg}}
\sphinxsetup{VerbatimBorderColor={named}{nbsphinx-code-border}}
\begin{sphinxVerbatim}[commandchars=\\\{\}]
\llap{\color{nbsphinxin}[ ]:\,\hspace{\fboxrule}\hspace{\fboxsep}}
\end{sphinxVerbatim}
}

\end{sphinxuseclass}
\end{sphinxuseclass}

\chapter{Indices and tables}
\label{\detokenize{index:indices-and-tables}}\begin{itemize}
\item {} 
\sphinxAtStartPar
\DUrole{xref,std,std-ref}{genindex}

\item {} 
\sphinxAtStartPar
\DUrole{xref,std,std-ref}{modindex}

\item {} 
\sphinxAtStartPar
\DUrole{xref,std,std-ref}{search}

\end{itemize}



\renewcommand{\indexname}{Index}
\printindex
\end{document}